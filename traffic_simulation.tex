\documentclass[9pt,a4paper,twoside]{article}
\usepackage[T2A]{fontenc}
\usepackage[utf8]{inputenc}
\usepackage[russian]{babel}

% Улучшенные шрифты для сглаживания
\usepackage{lmodern}        % Latin Modern - улучшенная версия Computer Modern
\usepackage{paratype}       % Шрифты ParaType для русского языка
\usepackage{microtype}      % Улучшенная микротипографика и сглаживание

\usepackage{hyphenat} % Для улучшенных переносов
\sloppy % Разрешает более свободное форматирование
\frenchspacing % Правильные пробелы после точек

\usepackage[
    paperwidth=140mm,
    paperheight=212mm,
    top=18mm,
    bottom=12mm,
    inner=12mm,
    outer=12mm
]{geometry}
\usepackage{indentfirst}
\usepackage{setspace}

\setstretch{1.0}
\setlength{\parindent}{0pt}
\setlength{\parskip}{2mm}

% Предотвращение висячих строк (меньше 3 строк внизу/вверху страницы)
\widowpenalties 3 10000 10000 0    % Запрет 1-2 строк вверху страницы
\clubpenalties 3 10000 10000 0     % Запрет 1-2 строк внизу страницы
\displaywidowpenalty=10000         % Для формул

% Настройка колонтитулов
\usepackage{fancyhdr}
\pagestyle{fancy}

% Настройка расстояний
\setlength{\headsep}{6mm}  % Уменьшенное расстояние между header и текстом
\setlength{\headheight}{18pt}  % Высота header для номера страницы и названия

% Очищаем колонтитулы
\fancyhf{}

% Настраиваем header
\fancyhead[LE]{\thepage\quad 17. Тише едешь — дальше будешь}  % Left on Even - слева на чётных
\fancyhead[RO]{17. Тише едешь — дальше будешь\quad\thepage}  % Right on Odd - справа на нечётных

% Убираем разделительные линии
\renewcommand{\headrulewidth}{0pt}
\renewcommand{\footrulewidth}{0pt}

\title{Моделирование движения на автостраде}
\author{Из классической книги по программированию}
\date{}

\begin{document}

\thispagestyle{empty}  % Убираем колонтитул с первой страницы

\setcounter{page}{95}

\vspace*{-15mm}  % Поднимаем заголовок вверх (звёздочка обязательна!)

{\large 17}

\vspace{1cm}

{\Large Тише едешь — дальше будешь,}

или...

{\large МОДЕЛИРОВАНИЕ ДВИЖЕНИЯ}

{\large НА АВТОСТРАДЕ}

\vspace{5mm}  % Отступ между заголовком и текстом

Энергетический кризис уже в своём начале привёл к снижению допустимой скорости движения на шоссе и автострадах в масштабах всей страны.
Большинство автомобилистов в длительных поездках не устанут повторять возражение.
Разумеется, теперь мы знаем, что снижение скорости ежегодно сберегает тысячи жизней и миллионы долларов.
Но мало кто из водителей понимает, что условное ограничение скорости в больших городах снижение скорости на самом деле ведёт к экономии времени.
Более того, парадокс формулируется так: если все тише едут, они скорее приедут.

Вспомните, как вы однажды «с ветерком» катили по шоссе, миль на 5 превышая допустимую скорость, хотя машин было много.
Внезапно все вокруг стали ползти, и вам тоже пришлось нажать на тормоза.
Затем последовала четверть, половина, а то и целая миля непрерывного ползти в окружении движения.
Наконец затор остался позади, и вы смогли вновь прибавить скорость.
Но всё это происходило без всякой видимой причины!
Что же нарушило плавность движения?

Для объяснения причины задержки необходимо привлечь гидродинамику.
Движущиеся по шоссе автомобили ведут себя во многом подобно частицам протекающей в трубе жидкости.
Если плотность и скорость частиц (автомобилей) велики, любая кратковременная задержка потока приведёт к возникновению ударной волны.
Ударная волна — это область очень высокой плотности; автомобили (или частицы) резко замедляются, попадая в эту область,
и затем снова, когда, преодолев довольно чётко очерченный ударный фронт, выходя из неё в область с гораздо более низкой плотностью.
Ударная волна продолжает существовать длительное время, медленно двигаясь навстречу потоку и медленно рассеиваясь.
Отметим, что рассеяние объясняется уменьшением плотности в ударной области и может быть ускорено, если водители заранее слегка притормозят, увидев впереди затор.

Было бы любопытно провести эксперимент на автостраде в часы пик, но, несомненно, пришлось бы привлечь не одну сотню машин.
Не лучше ли обойтись одной вычислительной машиной?
Рассмотрим прямой однорядный участок автострады длинной 5~миль, без перекрёстков.
Автомобили появляются на одном конце дороги, проезжают по ней и исчезают почему-то на другом конце.
Машины стремятся двигаться по дороге с постоянными скоростями (возможно, разными для разных машин).
Чтобы изучать ударные волны, будем вводить в эту транспортную благодать случайные замедления.

Для проведения эксперимента нужны генератор автомобилей и генератор возмущений.
В начале каждого эксперимента автострада пуста.
Запустите генератор автомобилей, который поместит машины на дорогу, придаст и скорость и выберет интервал до порождения следующего автомобиля.
Начальные скорости автомобилей подчиняются равномерному случайному распределению на отрезке от 50 до 80~миль~в~час,
а интервалы между порождениями также равномерно распределены на отрезке от 4~до~6~с.
Минимальное допустимое сближение составляет одну длину автомобиля (10 футов) на каждые 5~миль~в~час скорости передней машины.
Когда автомобиль приближается к идущей впереди машине на упомянутое допустимое расстояние, он начинает притормаживать, пока скорости не сравняются, теряя по одной миле в час за секунду.
Затем передний автомобиль начинает резко замедляться, идущий сзади выжидает 0,2~с и затем тормозит, снижая каждую секунду свою скорость на 15~миль~в~час.
В результате может произойти авария, которой и закончится эксперимент.

Собственно эксперимент состоит в заполнении дороги машинами, введении искусственного замедления и наблюдении результата.
Начните запускать машины на дорогу; продолжайте делать это, пока не пройдёт 2~минуты (модельного времени) с момента прохождения заданного участка дороги первым автомобилем.
Затем, не прекращая запускать машины, выберите автомобиль, который раньше всех пересечёт отметку в 4~мили,
сбросьте с его скорости 0, 10, 20, 30, 40 или 50~миль~в~час, держите на новой скорости 100~секунд, после чего придайте ему ускорение 5~миль~в~час~за~секунду,
пока автомобиль не наберёт свою первоначальную скорость (максимально допустимую скорость).
Продолжите эксперимент ещё 5~минут после того, как виновник затора начал замедляться, и попытайтесь количественно оценить длину и происхождение волн за это время.
Получится следующий результат эксперимента: машины, следующие за виновником, также могут ускоряться на 5~миль~в~час~за~секунду, если дорога перед ними освободождается.
Проведите эксперимент несколько раз для каждого значения замедления.
Если произойдёт авария, все машины, находящиеся позади автоматически остановятся и не смогут пройти заданный участок дороги.
В аварию может попасть не сам виновник, а машины, идущие сзади.

\textit{Тема.} Напишите программу, позволяющую провести эксперимент с ударной волной на автостраде.
Все исходные данные представлены одним числом — скоростью поворота эксперимента для каждого уменьшения скорости.
Обязательная часть выводимой информации — среднее количество машин, прошедших участок дороги после каждого искусственного замедления.
Но для отладки и лучшего понимания физического поведения системы полезно вывести дополнительную информацию.
В частности, несколько «моментальных снимков» дороги, вероятно, позволят почувствовать ситуацию лучше, чем любое количество статистики.
Если в вашем распоряжении имеется хорошее графическое устройство — интерактивное или микрофильмовое — то серия моментальных снимков составит анимацию о дороге.

\textit{Указания исполнителю.} Наиболее трудным в предлагаемой задаче является отслеживание всех автомобилей на дороге\footnote{Дополнительную
трудность вызывает использование традиционных англо-американских мер. Однако сделано это умышленно, и мы должны выдавать результаты в тех же единицах.
Если бы скорость измерялась в м/с т.~е. в дюймах в день, было бы ещё хуже...}.
Можно организовать цикл и примерно через одну сотую — одну десятую секунды модельного времени подправлять положение и скорости каждого автомобиля.
Если интервал достаточно мал, заметного накопления ошибок не произойдёт, а выглядеть программа будет красиво — как система нескольких циклов.
Однако при использовании метода пошаговой фиксации цикл может выполняться слишком большое число раз.
В нашем случае эксперимент продлится примерно 12~минут модельного времени, в каждый момент на дороге будет около 90~машин, и,
даже если выбрать шаг цикла в одну десятую секунды, потребуется примерно 1200~циклов, или всего 650 000~операций с отдельными автомобилями.
Если программа тратит много времени на продвижение одного автомобиля, эксперимент слишком затянется.
Положение можно подправить, варьируя интервал в зависимости от дорожной обстановки.

Другой подход состоит в том, чтобы подправлять положение автомобилей только в моменты критических событий.
При таком подходе заводятся часы всех событий, ожидаемых в недалеком будущем; например, запускается или исчезает автомобиль, одна машина догоняет другую,
прошло две минуты с момента исчезновения первого автомобиля, пора вновь ускорять автомобиль-виновник.
Головным элементом списка событий всегда должно быть ближайшее событие,
которое в целом не обязано быть упорядоченным — его можно представлять и как очередь с приоритетами, и как кучу.
В основном цикле от списка отделяется головной элемент, время текущего события сообщается с ним,
все автомобили устанавливаются в позиции, соответствующие новому времени, запоминаются все события,
которые следует \textit{планировать}, эти события вставляются в список и список переупорядочивается,
чтобы ближайшее событие оказалось в голове. \textit{Достоинство моделирования методом критических событий} в том, что порой довольно долго, 4—5~секунд, ничего не происходит.
\textit{Сэкономленное время} можно употребить на более сложную обработку списка событий.

\textit{Инструментовка.} Для решения этой задачи естественно воспользоваться языками моделирования, такими, как Симскрипт или Симула.
Если они недоступны, подойдет любой процедурный язык.
Независимо от метода моделирования существенным подспорьем будут хорошие структуры данных для представления информации об автомобилях и для реализации очереди событий.

\textit{Длительность исполнения.} Одному исполнителю на 3 недели; ещё неделя на изготовление фильма.

\textit{Развитие темы.} Строго говоря, в предложенной задаче не изучается ситуация, описанная в нескольких первых абзацах.
Вместо выяснения того, что происходит с ударной волной при различных средних скоростях движения на автостраде, в эксперименте рассматривались удары различной силы.
Проделайте всё ещё раз, взяв диапазон начальных скоростей от 40 до 50 или от 60 до 70~миль в час.
Попробуйте для нескольких нормальное распределение вместо равномерного.
Поварьируйте законы торможения и ускорения.
Иными словами, изучите влияние всех параметров, а не только одного, выбранного нами.

\subsection*{Литература}

\noindent Herman, Gardels (Herman R., Gardels K.). Vehicular Traffic Flow. \textit{Scientific American}, pp. 35—43, December 1963.

Авторы описали проведение нескольких физических экспериментов над движением транспорта и развитие математической теории.
Конечно, использовавшийся ими Голландский туннель в Нью-Йорке для большинства из нас недоступен.
Если вам интересно проследить за работами по этой тематике после 1963~г., научитесь пользоваться \textit{Science Citation Index} или другими библиографическими средствами,
помогающими довести старую информацию до наших дней.

\end{document}
